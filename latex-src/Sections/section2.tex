% now that we have justified why we can use types to do 
% formal verified programming
% lets see what dependent types are in theory and in practice 

% Section 2
\section{Dependent Types theory}


\begin{frame}
\frametitle{Dependent Types Theory MVPs: $\Pi$-type \& $\Sigma$-type}
\textcolor{red}{Dependent types are families of types which vary over the elements (terms) of some other type.}
\\ \medskip
The two principal instances of dependent types are \textbf{dependent functions} ($\Pi$-types ) and \textbf{dependent pairs} ($\Sigma$-type).
\\ \medskip
The terms of a $\Pi$-type are type functions whose codomain type can vary depending on the element of the domain to which the function is applied
\\ \medskip
$\Sigma$-types are a generalization of product types in which the type of the second component of the pair is allowed to vary dependending on the value of the first component.
\end{frame}


\begin{frame}
\frametitle{Dependent Function ($\Pi$-type) - Notation}
Dependent function types are similar to the type of an indexed family of sets.\\ \medskip
Formally, given a type $\pmb{A:\mathcal{U}}$ in a universe of types $\pmb{\mathcal{U}}$, \\ 	
one may have a family of types $\pmb{B : A \rightarrow \mathcal{U}}$ which assigns to each term $\pmb{a: A}$ a type $\pmb{B(a): \mathcal{U}}$.\\ \medskip
We say that type $\pmb{B(a)}$ depends on the term a.\\ \medskip
Dependent function types are written as: $\pmb{\prod_{x:A} B(x)}$ \\ \medskip
If $\pmb{B : A \rightarrow \mathcal{U}}$ is a constant function, then the dependent function type is equivalent to an ordinary function type.\\ \medskip
It follows that $\Pi$-type is the generalization of function types.
\end{frame}


\begin{frame}
\frametitle{Dependent Pair ($\Sigma$-type) - Notation}
The dual of the dependent function type is the dependent pair type. In set theory it corresponds to an indexed sum (disjoint union) of types.\\ 
\medskip 
In the universe of types $\pmb{\mathcal{U}}$ there is a type $\pmb{A:\mathcal{U}}$ and a family of types $\pmb{B : A \rightarrow \mathcal{U}}$\\
then there is a dependent pair type
$\pmb{\sum_{x:A}B(x)}$.\\ \medskip
The dependent pair type captures the idea of a pair where the type of the second term is dependent on the value of the first.\\
If $\pmb{(a,b):\sum_{x:A}B(x)}$
then $\pmb{a:A}$ and $\pmb{b:B(a)}$.\\ \medskip
If B is a constant function, then the dependent pair type is the ordinary (Cartesian) product type $\pmb{A×B}$
\end{frame}
%A dependent pair may be used to pair an array with its length in a type-safe way.

 
\begin{frame}
\frametitle{$\Pi$-type and $\Sigma$-type examples}
\begin{itemize}
	\item[$\Pi$)] let $\pmb{Vect(\mathbb{R}, k)}$ be the type of k-tuples of real numbers, then
	\[\pmb{\prod_{n:\mathbb{N}}Vect(\mathbb{R}, n)}\]
	would be the type of a dependent function which, given a natural number n, returns the type of tuples of reals of size n.
	
	\item[$\Sigma$)] let $\pmb{OrdPair(a,b)}$ be the type of pairs where the first element is less than the second element, then \[\pmb{\sum_{(a,b) : \mathbb{N} \times \mathbb{N}} \mathtt{Less}(a,b)}\]
	 would be the type of a dependent pair which couples two natural numbers a and b with a proof of $\pmb{a<b}$

\end{itemize}
\end{frame}

\begin{frame}
\frametitle{$\Pi$-type and $\Sigma$-type can be interpreted in two ways}
By \textbf{propositions-as-types} assumption
\begin{itemize}
	\item \textcolor{red}{$\Pi$-types can be interpreted as universal quantifiers $\forall$}.
	\item \textcolor{red}{$\Sigma$-types can be interpreted as existential quantifiers $\exists$.}
\end{itemize}
So every dependent type can be read either as a proposition or as a set. This gives us two readings:
\begin{enumerate}
	\item
	a dependent type $\pmb{P:\mathbb{N} \rightarrow \mathcal{U}}$ is like a family of sets $\pmb{P(0), P(1), P(2), \ldots}$
	\item
	a dependent type $\pmb{P:\mathbb{N} \rightarrow \mathcal{U}}$ is a property of natural numbers where we think of $\pmb{P(n)}$ as the proofs that $\pmb{n}$ has property $\pmb{P}$
\end{enumerate}
\end{frame}

% Dependent type systems are heavily used for software certification.\\